% terms
\newglossaryentry{aliasing} {
  name=aliasing,
  description={artifacts caused when sampling textures and performing calculations, often manifested as rough edges}
}

\newglossaryentry{float} {
  name={float (\textmd{also} floating point)},
  text=float,
  description={system of describing an arbitrary decimal number, rational or irrational, with a high amount of precision; compare to fixed point, where the precision of the fractional and whole components is predetermined}
}

\newglossaryentry{floating point} {
	name=floating point,
	description={\emph{See} \gls{float}}
}

\newglossaryentry{OpenGL} {
  name=OpenGL,
  description={industry standard library used for interfacing with \glspl{graphics accelerator}. Developed by SGI in the 1990s}
}

\newglossaryentry{graphics accelerator} {
  name=graphics accelerator,
  description={a dedicated piece of hardware (usually in the form of a plug-in card or chip) that is optimized to perform many complex calculations in hardware, in parallel}
}

\newglossaryentry{frame rate} {
  name=frame rate,
  description={the rate at which the graphics output of an application (i.e. its frame) is updated}
}

\newglossaryentry{refresh rate} {
  name=refresh rate,
  description={the rate at which the video hardware updates the display: the maximum rate at which the screen can be updated}
}

\newglossaryentry{memory bandwidth} {
  name=memory bandwidth,
  description={the maximum continuous rate at which a \gls{graphics accelerator} can read data from its memory. Usually specified in transfers per second}
}

\newglossaryentry{shader} {
  name=shader,
  description={code that is executed on the GPU, performing a variety of processing tasks for \glspl{texel} to be output. A \gls{GPU} may have hundreds of \glspl{compute unit}, each of which can run a shader over a given set of data}
}

\newglossaryentry{compute unit} {
  name=compute unit,
  description={mathematics processing unit which is used to execute a single \gls{shader} over many  data sets in parallel}
}

\newglossaryentry{texel} {
  name=texel,
  description={a single pixel on a texture, usually identified by either a 2D or a 3D point}
}

\newglossaryentry{graphics pipeline} {
  name=graphics pipeline,
  description={all components of a \gls{GPU} connected end-to-end, going from input data to a \gls{texel} on the screen}
}

\newglossaryentry{pipeline stall} {
  name=pipeline stall,
  description={occurs when a 'bubble' is allowed to enter the \gls{graphics pipeline}: either because the \gls{GPU} must wait on data or calculations, or if a significant state change (such as changing \gls{shader} programs) takes place}
}

\newglossaryentry{deferred shading} {
  name=deferred shading,
  description={technique in which all models are rendered into an intermediary buffer, called the \gls{G Buffer}, before lighting calculations are applied}
}

\newglossaryentry{forward shading} {
  name=forward shading,
  description={technique in which all lighting calculations are performed when a texel is rendered, regardless of whether it shows up in the final output or not}
}

\newglossaryentry{multisampling} {
  name=multisampling,
  description={sampling the same object several times, possibly at different coordinates, then producing a single output value}
}

\newglossaryentry{supersampling} {
  name=supersampling,
  description={sampling the the same object at a much higher resolution than is needed, in order to produce smoother outputs and combat aliasing artifacts}
}

\newglossaryentry{G Buffer} {
  name=geometry buffer,
  description={a buffer that stores depth, specular highlights, albedo, and surface normals}
}

\newglossaryentry{api-def} {
  name=application programming interface,
  description={a well-defined standard describing the way in which applications interface with third-party libraries}
}

\newglossaryentry{hdr-def} {
  name=high dynamic range,
  description={technique in which all color values are not limited to a finite display range, but are instead represented as \gls{floating point} values, then normalized into the display range later}
}

\newglossaryentry{ldr-def} {
  name=low dynamic range,
  description={heavily compressed range of colour values that be accurately reproduced by the display device}
}

\newglossaryentry{bloom} {
  name=bloom,
  description={a blur surrounding edges of very brightly lit objects; usually implemented in conjunction with \gls{HDR}}
}

\newglossaryentry{shadow mapping} {
  name=shadow mapping,
  description={a method used to cast shadows, by rendering the scene from the viewpoint of a light source, then transforming the resultant depth when rendering lighting}
}

\newglossaryentry{ambient occlusion} {
  name=ambient occlusion,
  description={technique used to calculate how exposed each particular point in a scene is to ambient lighting}
}

\newglossaryentry{fxaa-def} {
  name=fast approximate anti-aliasing,
  description={a post-processing technique that reduces sharp edges on objects caused by \gls{aliasing} through edge detection}
}

\newglossaryentry{depth test} {
  name=depth test,
  description={step in the rendering pipeline, where the depth value of a texel is compared to that of a depth buffer to decide whether the texel is drawn or not}
}

\newglossaryentry{Blinn-Phong reflection model} {
  name=Blinn-Phong reflection model,
  description={model that approximates the way in which light is reflected off of a surface in the most realistic way}
}

\newglossaryentry{skybox} {
  name=skybox,
  description={static background that is rendered in areas where no objects are drawn to provide a sense of scale and define the environment}
}

\newglossaryentry{rgb-def} {
  name=red green blue,
  description={the three primary colours used in computer displays to produce any possible colour; also known as additive mixing}
}

\newglossaryentry{specular} {
  name=specular,
  description={bright spots of lights on an object: i.e. the light that is reflected off of an object, based on the angle of the light source relative to the viewer}
}

\newglossaryentry{diffuse} {
  name=diffuse,
  description={solid colour of an object, i.e. light that is reflected off of an object, regardless of the viewer's orientation toward it}
}

\newglossaryentry{tone mapping} {
  name=tone mapping,
  description={process of converting \gls{HDR} colours to display (\gls{RGB}) colours}
}

\newglossaryentry{white point} {
  name=white point,
  description={a vector that describes where in a colour space white is, in essence defining its origin}
}

\newglossaryentry{Reinhard tone mapping} {
  name=Reinhard tone mapping,
  description={\gls{tone mapping} algorithm that evenly spreads out the \gls{LDR} values to \gls{HDR} values, then balancing them}
}

\newglossaryentry{bits per pixel} {
  name=bits per pixel,
  description={number of bits required to represent a single pixel in video memory}
}